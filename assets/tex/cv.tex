% article class because we want to fully customize the page and not use a cv template
\documentclass[letterpaper,12pt]{article}

%----------------------------------------------------------------------------------------
%	FONT
%----------------------------------------------------------------------------------------

% % fontspec allows you to use TTF/OTF fonts directly
% \usepackage{fontspec}
% \defaultfontfeatures{Ligatures=TeX}

% % modified for ShareLaTeX use
% \setmainfont[
% SmallCapsFont = Fontin-SmallCaps.otf,
% BoldFont = Fontin-Bold.otf,
% ItalicFont = Fontin-Italic.otf
% ]
% {Fontin.otf}

%----------------------------------------------------------------------------------------
%	PACKAGES
%----------------------------------------------------------------------------------------
\usepackage{url}
\usepackage{parskip}

%other packages for formatting
\RequirePackage{color}
\RequirePackage{graphicx}
\usepackage[usenames,dvipsnames]{xcolor}
\usepackage[scale=0.9]{geometry}

%tabularx environment
\usepackage{tabularx}

%for lists within experience section
\usepackage{enumitem}

% centered version of 'X' col. type
\newcolumntype{C}{>{\centering\arraybackslash}X} 

%to prevent spillover of tabular into next pages
\usepackage{supertabular}
\usepackage{tabularx}
\newlength{\fullcollw}
\setlength{\fullcollw}{0.47\textwidth}

%custom \section
\usepackage{titlesec}				
\usepackage{multicol}
\usepackage{multirow}

%CV Sections inspired by: 
%http://stefano.italians.nl/archives/26
\titleformat{\section}{\Large\scshape\raggedright}{}{0em}{}[\titlerule]
\titlespacing{\section}{0pt}{10pt}{10pt}

%for publications
\usepackage[style=authoryear, maxbibnames=99, uniquename=false, dashed=false, sorting=ydnt]{biblatex}

%Setup hyperref package, and colours for links
\usepackage[unicode, draft=false]{hyperref}
\definecolor{linkcolour}{rgb}{0,0.2,0.6}
\hypersetup{colorlinks,breaklinks,urlcolor=linkcolour,linkcolor=linkcolour}
\addbibresource{../../_bibliography/papers.bib}
\addbibresource{../../_bibliography/posters.bib}
\setlength\bibitemsep{1em}

% Custom name format to handle different cases
\DeclareNameFormat{custom}{%
  % Check if the current author is the first one
  \ifnumequal{\value{listcount}}{1}
    {\namepartfamily \space \namepartgiven}
    {}%
  
  % Check if the current author is the second one
  \ifnumequal{\value{listcount}}{2}
    {%
      \ifnumgreater{\value{liststop}}{2}
        {\ifboolexpr{ test {\ifnameequal{#1}{Isaac G.}{Smith}} }
          {\namepartfamily \space \namepartgiven \addcomma\space \bibstring{etal}}
          {\namepartfamily \space \namepartgiven}}%
        {\namepartfamily \space \namepartgiven}%
    }
    {}%
  
  % Check if the current author is beyond the second
  \ifnumgreater{\value{listcount}}{2}
    {%
      \ifboolexpr{ test {\ifnameequal{#1}{Isaac G.}{Smith}} }
        {%
          \ifnumless{\value{listcount}}{\value{liststop}-1}
            {\namepartfamily \space \textellipsis \space \namepartgiven}
            {\namepartfamily \space \textellipsis \space \namepartgiven \addcomma\space \bibstring{etal}}%
        }
        {}%
    }
    {}%
}

% Apply the custom name format to the author field globally
\AtEveryBibitem{\DeclareNameAlias{author}{custom}}

%for social icons
\usepackage{fontawesome5}

%debug page outer frames
%\usepackage{showframe}

%----------------------------------------------------------------------------------------
%	BEGIN DOCUMENT
%----------------------------------------------------------------------------------------
\begin{document}

% non-numbered pages
\pagestyle{empty} 

%----------------------------------------------------------------------------------------
%	TITLE
%----------------------------------------------------------------------------------------

% \begin{tabularx}{\linewidth}{ @{}X X@{} }
% \huge{Your Name}\vspace{2pt} & \hfill \emoji{incoming-envelope} email@email.com \\
% \raisebox{-0.05\height}\faGithub\ username \ | \
% \raisebox{-0.00\height}\faLinkedin\ username \ | \ \raisebox{-0.05\height}\faGlobe \ mysite.com  & \hfill \emoji{calling} number
% \end{tabularx}

\begin{tabularx}{\linewidth}{@{} C @{}}
\Huge{Isaac G. Smith} \\[7.5pt]
\href{https://github.com/isaacgsmith}{\raisebox{-0.05\height}\faGithub\ isaacgsmith} \ $|$ \ 
% \href{https://linkedin.com/in/username}{\raisebox{-0.05\height}\faLinkedin\ username} \ $|$ \ 
\href{https://isaacgsmith.github.io}{\raisebox{-0.05\height}\faGlobe \ isaacgsmith.github.io} \ $|$ \ 
\href{https://orcid.org/0000-0003-0440-3918}{\raisebox{-0.05\height}\faOrcid \ orcid.org/0000-0003-0440-3918} \\ 
\href{mailto:isaacsmi@weizmann.ac.il}{\raisebox{-0.05\height}\faEnvelope \ isaacsmi@weizmann.ac.il} \ $|$ \ 
\href{tel:+972-053-367-5546}{\raisebox{-0.05\height}\faMobile \ +972-053-367-5546} \\
\end{tabularx}


%----------------------------------------------------------------------------------------
%	EDUCATION
%----------------------------------------------------------------------------------------
\section{Education}
\begin{tabularx}{\linewidth}{@{}l X@{}}	

2024 - present & \textbf{Weizmann Institute of Science (WIS)} \\ 

2020 - 2024 & \textbf{Michigan State University (MSU)} \hfill \normalsize (GPA: 4.0/4.0) \\
 & B.S. in Physics, College of Natural Science, Honors College \hfill \\
 & B.S. in Mathematics, Advanced, College of Natural Science, Honors College \hfill \\
 & Minor in Music (vocalist), College of Music \hfill \\

\end{tabularx}


%----------------------------------------------------------------------------------------
% RESEARCH EXPERIENCE
%----------------------------------------------------------------------------------------
\section{Research Experience}
\begin{tabularx}{\linewidth}{ @{}l r@{} }

\textbf{Research Assistant, Facility for Rare Isotope Beams, MSU} & \hfill Nov 2022 - Jul 2024 \\[3.75pt]
\multicolumn{2}{@{}X@{}}{
\begin{minipage}[t]{\linewidth}
    \begin{itemize}[nosep,after=\strut, leftmargin=1em, itemsep=3pt]
        \item[--] Developed and implemented a finite-temperature formalism for the IMSRG many-body solver.
        \item[--] Analyzed data from the finite-temperature IMSRG using an exactly-solvable schematic model.
        \item[--] Studied the effect of temperature on the stability of calcium isotopes. 
    \end{itemize}
    \end{minipage}
} \\[5pt]

\textbf{Research Assistant, TARDIS Collaboration, MSU} & \hfill Sep 2020 - Aug 2023 \\[3.75pt]
\multicolumn{2}{@{}X@{}}{
\begin{minipage}[t]{\linewidth}
    \begin{itemize}[nosep,after=\strut, leftmargin=1em, itemsep=3pt]
        \item[--] Wrote an extensive physics walkthrough for the TARDIS radiative transfer code.
        \item[--] Participated in MSU’s Engineering Summer Undergraduate Research Experience program.
        \item[--] Developed the STARDIS stellar radiative transfer code, a companion code to TARDIS.
    \end{itemize}
    \end{minipage}
} \\

\end{tabularx}


%----------------------------------------------------------------------------------------
% TEACHING AND MENTORING
%----------------------------------------------------------------------------------------
\section{Teaching and Mentoring}
\begin{tabularx}{\linewidth}{ @{}l r@{} }

\textbf{Learning Assistant for Calculus II, MSU} & \hfill Jan 2024 - Apr 2024 \\[3.75pt]
\multicolumn{2}{@{}X@{}}{
\begin{minipage}[t]{\linewidth}
    \begin{itemize}[nosep,after=\strut, leftmargin=1em, itemsep=3pt]
        \item[--] Taught weekly recetations for two Calculus II classes.
        \item[--] Tutored calculus students in MSU's Math Learning Center.
        \item[--] Graded quizzes and exams.
    \end{itemize}
    \end{minipage}
} \\[5pt]

\textbf{Mentor, TARDIS Collaboration, MSU} & \hfill May 2021 - Aug 2023 \\[3.75pt]
\multicolumn{2}{@{}X@{}}{
\begin{minipage}[t]{\linewidth}
    \begin{itemize}[nosep,after=\strut, leftmargin=1em, itemsep=3pt]
        \item[--] Mentored seven students in contributing to the TARDIS collaboration through the TARDIS high
school program, professorial assistantships, or Google’s Summer of Code.
        \item[--] Led weekly meetings discussing physics concepts that are used in the TARDIS code.
    \end{itemize}
    \end{minipage}
} \\[5pt]

\textbf{Tutor} & \hfill Sep 2018 - June 2024 \\[3.75pt]
\multicolumn{2}{@{}X@{}}{
\begin{minipage}[t]{\linewidth}
    \begin{itemize}[nosep,after=\strut, leftmargin=1em, itemsep=3pt]
        \item[--] Tutored over 20 students in subjects including physics, calculus, and biology.
    \end{itemize}
    \end{minipage}
} \\

\end{tabularx}


%----------------------------------------------------------------------------------------
%	PUBLICATIONS
%----------------------------------------------------------------------------------------
\section{Publications}
\begin{refsection}[../../_bibliography/papers.bib]
\nocite{*}
\printbibliography[heading=none, nameformat=highlight]
\end{refsection}


%----------------------------------------------------------------------------------------
%	PRESENTATIONS
%----------------------------------------------------------------------------------------
\section{Presentations}
\begin{refsection}[../../_bibliography/posters.bib]
\nocite{*}
\printbibliography[heading=none, nameformat=highlight]
\end{refsection}


%----------------------------------------------------------------------------------------
%	PROJECTS
%----------------------------------------------------------------------------------------
\section{Major Projects and Unpublished Work}
\begin{tabularx}{\linewidth}{ @{}l r@{} }

\textbf{The Geometric Formulation of Classical Physics} & \hfill \\[3.75pt]
\multicolumn{2}{@{}X@{}}{My undergraduate thesis in mathematics, which details the relationship between symplectic geometry and classical mechanics, as well as the relationship between measure theory, contact geometry, statistical mechanics, and thermodynamics.}  \\[5pt]

\textbf{STARDIS Radiative Transfer Code} & \hfill \href{https://github.com/tardis-sn/stardis}{Link to Repository} \\[3.75pt]
\multicolumn{2}{@{}X@{}}{I made major contributions to the early development of the STARDIS radiative transfer code.}  \\[5pt]

\textbf{TARDIS Documentation} & \hfill \href{https://tardis-sn.github.io/tardis/}{Link to Documentation} \\[3.75pt]
\multicolumn{2}{@{}X@{}}{I designed comprehensive, interactive documentation for the TARDIS radiative transfer code, and was the main author for the physics walthrough.}  \\

\end{tabularx}


%----------------------------------------------------------------------------------------
%	HONORS AND AWARDS
%----------------------------------------------------------------------------------------
\section{Honors and Awards}

\begin{tabularx}{\linewidth}{@{}l X@{}}	
2024 & Carl L. Foiles Award, MSU (top graduating physics student) \\
2024 & Board of Trustees Award, MSU (4.0 GPA) \\
2024 & MSU Integration Bee Third Place \\
2023 & Jeffrey R. Cole Honors College Research Fund, MSU \\
2021, 2023 & Lawrence W. Hantel Fellowship, MSU (physics research award) \\
2023 & Nominee, Rhodes Scholarship, MSU \\
2023 & Nominee, Marshall Scholarship, MSU \\
2022 & L.C. Plant Mathematics Award, MSU \\
2021, 2022 & NumFOCUS Small Development Grant \\
2020 & Alumni Distinguished Scholar, MSU (MSU's top merit scholarship) \\
2020 & National Merit Scholar \\
2020-2024 & Dean's List, MSU (all semesters) \\
\end{tabularx}


%----------------------------------------------------------------------------------------
%	SKILLS
%----------------------------------------------------------------------------------------
\section{Skills}

\begin{tabularx}{\linewidth}{@{}l X@{}}
Proficient in Python, C, C++, Git, and Linux \\
Intermediate level in Hebrew \\  
\end{tabularx}

\vfill
\center{\footnotesize Last updated: \today}

\end{document}
